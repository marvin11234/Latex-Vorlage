\onehalfspacing
\section{Fazit / Ausblick}
In diesem Kapitel wird rückwirkend auf das Projekt eingegangen. Der Ablauf des Projekts wird kritisch reflektiert. Im weiteren Verlauf des Kapitels wird auf den zukünftigen Verlauf des Projekts eingegangen.
\subsection{Fazit}
Ziel dieser Arbeit war es eine Testsuite für die anstehende Portierung des SKx zu erstellen. Die Testsuite soll den Entwickler von Softwaretests bei seiner Aufgabe unterstützen und somit den Prozess beschleunigen. Die Testsuite sollte in das Projekt Automated Testing @ GBC07 integriert werden.
Zu Beginn des Projektes wurde ein Überblick über die verschiedenen Softwaretests geschaffen. Als Herausforderung hierbei stellte sich heraus, dass es noch keine Dokumentation der SKx Softwaretests gibt. Aus diesem Grund wurde auf die Dokumentation der Softwaretests am SEY zurückgegriffen. Die verschiedenen Softwaretests wurden nach definierten Kriterien analysiert und geprüft welche von ihnen sich mit der Suite abdecken lassen. Hierbei lag das Augenmerk auf der Automatisierbarkeit der Tests. Die in diesem Arbeitsschritt identifizierten Tests wurden im Anschluss geclustert und herausgearbeitet welche Hardware Komponenten verwendet werden. Aus den erarbeiteten Informationen wurde Anforderungen an das Projekt definiert und diese im Anschluss mit den Kunden abgestimmt. Die Requirements wurden hierbei in funktional und nicht-funktionale unterteilt. Die Analyse- und Konzeptionsphase zog sich real deutlich länger, als ursprünglich geplant. Grund hierfür war die aufwendige Analyse der Testfälle. In der folgenden Konzeptionsphase wurden zuerst die bisher vorhandenen Softwarekomponenten (Motor-, Hiperfacetreiber etc.) analysiert. Hier lag das Augenmerk darauf, zu identifizieren welche Teile der bisherigen Software eventuell wiederverwendet werden können und welche Architektur bisher verwendet wurde. Im zweiten Teil der Kozeptionierung erfolgte die Erstellung einer entsprechenden neuen. Hierfür wurden verschiedene Architekturmodelle (MVC, Pipe-Filter, Schichtenarchitektur) verglichen und das am Besten geeignete Modell ausgewählt. Das Ergebnis der Konzeption ist eine vier-Schichtarchitektur. Im Zuge der Konzeptionsphase wurde darüber hinaus eine wirtschaftliche Betrachtung des Projekts durchgeführt. Diese gestaltet sich jedoch schwierig, da sich nur schwer abschätzen lässt, welchen realen finanziellen Vorteil das Projekt in Zukunft bringen wird.
In der folgenden Implementierungsphase wurde ein Teil des Konzepts im Rahmen eines Proof of Concept umgesetzt. Die Programmierung erfolgte hierbei in der durch SICK entwickelten Programmiersprache ITE. Durch die Verwendung von ITE kam es bereits während der Konzeption zu verschiedenen Herausforderungen, da sich mit ihr gewisse Konstrukte wie zum Beispiel Interfaces nicht wie in andern objektorientierten Sprachen abbilden lassen. Durch diese Herausforderung konnten nicht alle architektonischen Probleme mit bekannten Entwurfsmustern gelöst werden.
Nach der prototypischen Implementierung wurden im Rahmen der Validierung drei verschiedene Testfiles  und die Auswirkung der Testsuite auf die Anzahl der benötigten Codezeilen untersucht. Trotz des Umstandes, dass es sich beim Proof of Concept lediglich um eine Teilimplementierung handelt, konnten hier bereits passable Ergebnisse erzielt werden. Weiterhin wurde das Ergebnis des Projekts mit den definierten Anforderungen abgeglichen, wobei diese weitestgehend erfüllt wurden.
\subsection{Ausblick}
Nachdem es sich im Rahmen dieser Arbeit nur um eine Proof of Concept Implementierung handelt, wird die tatsächliche Implementierung im Rahmen des Portierungsprojekt voraussichtlich fortgeführt werden. Hierbei kann davon ausgegangen werden, dass die in der Arbeit erstellten Konzepte und die darin enthaltene Architektur umgesetzt wird. Die bisher entwickelten Softwarekomponenten können hierfür weiterverwendet werden. Ein Refactoring, sowie ausführliche Tests sollten jedoch noch durchgeführt werden, um den Code zu verifizieren. Von Vorteil wäre eine enge Zusammenarbeit mit dem für die Portierung zuständigen Softwareentwickler. Nach Erstellung des konkreten Testplans durch den Entwickler müssen eventuell noch kleinere Änderungen an den \dq UserFunctions\dq~vorgenommen werden. Jedoch ist dies aufgrund der gewählten Architektur einfach möglich. Weiterhin muss die Logging Funktion, sowie die Ansteuerung der verschiedenen Netzteile implementiert werden. Durch die gewählte Architektur ist das Projekt auch für weitere Hardware, oder Softwareschnittstellen erweiterbar. So können zum Beispiel sehr schnell und komfortabel weitere Motoren eingebunden werden.  