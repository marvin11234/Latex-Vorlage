\onehalfspacing
\section{Evaluation}
In diesem Kapitel wird die Evaluation des Projekts beschrieben. Die Evaluation gliedert sich in zwei Bestandteile. Zuerst wird eine Nutzenanalyse durchgeführt. Mit dieser soll geprüft werden, welchen realen Nutzen das Projekt im späteren Einsatz liefert. Im zweiten Schritt wird das Ergebnis der Arbeit mit den erstellten Anforderungen aus Kapitel \dq \nameref{sec:Anforderungsdefinition}\dq~abgeglichen. 
	\subsection{Nutzen Analyse}
	Um zu bestimmen, welchen Nutzen die Einführung der Suite auf die Entwicklung eines Tests hat, werden zwei Kriterien verwendet. Als Erstes wird verglichen wie viele Zeilen Code durch die Verwendung der Testsuite im eigentlichen Testfile eingespart werden können. 
	Um zu bestimmen wie viele \ac{LOC} durch den Einsatz der Suite eingespart werden können, werden verschiedene bisherige Testfälle betrachtet. Hierbei handelt es sich um die Testfälle, welche in Tabelle \dq \nameref{tab:LOCTestfall}\dq~aufgeführt sind. 
	
\begin{table}[h]
\begin{center}
\begin{tabularx}{\textwidth}{|X|X|c|}
\hline
Name & Geräte & \ac{LOC} \\
\hline
CriticalSpeed & Faulhaber Motor & 659 \\
\hline
ExtendedFuncHFSinCos & Faulhaber Motor, Hiperface & 865 \\
\hline
MultiturnAngleVerification & Faulhaber Motor & 749 \\
\hline
\end{tabularx}
\end{center}
\caption{Auswertung \ac{LOC} pro Testfall \label{tab:LOCTestfall}}
\end{table}
In Tabelle \dq \nameref{tab:einsparung}\dq~sind die Testfälle, sowie deren \ac{LOC} bei Verwendung der Testsuite dargestellt:\newline
\begin{table}[h]
\begin{center}
\begin{tabularx}{\textwidth}{|X|c|c|c|}
\hline
Name & \ac{LOC} & Einsparung & Einsparung \% \\
\hline
CriticalSpeed & 659 & ca. 50 & 7,59 \\
\hline
ExtendedFuncHFSinCos & 865 & ca. 80 & 9,24 \\
\hline
MultiturnAngleVerification & 749 & ca. 35 & 4,67 \\
\hline
\end{tabularx}
\end{center}
\caption{Auswertung Einsparung \label{tab:einsparung}}
\end{table}

Zu erkennen ist hierbei, dass es in allen betrachteten Testfällen zu einer Reduzierung der \ac{LOC} kommt. Hierbei ist jedoch anzumerken, dass es sich bei den drei Testfällen nicht um Testfälle für den SKx handelt. Bei Testfällen für den SKx ist mit einer deutlich höheren Einsparung zu rechnen.
Der Hauptvorteil welcher sich durch die Anwendung der Suite ergibt ist jedoch eher im Bereich des Clean Code zu sehen. Durch die Suite wird die Verwendung der verschiedenen Hardwarekomponenten standardisiert und ausgelagert, was zu einem übersichtlicheren Testfile führt. Auch ist die Komplexität für den Entwickler selbst deutlich geringer, da dieser sich nicht mit der expliziten Implementierung zum Beispiel des Motortreibers auseinandersetzen muss.
	\subsection{Abgleich der Anforderungen}
	Die Anforderungen welche in \dq \nameref{sec:Anforderungsdefinition}\dq~definiert sind, werden hier mit der realen Implementierung abgeglichen. In den Tabellen \dq \nameref{tab:AbgleichF}\dq~und \dq  \nameref{tab:AbgleichNF}\dq~sind die Ergebnisse aufgeführt. Hierbei wird jeweils zwischen Kriterien \dq erfüllt\dq, \dq teilweise erfüllt\dq~und \dq nicht erfüllt\dq~unterschieden. Alle Anforderungen werden grundlegend erfüllt. Ausnahme hierbei sind die Anforderungen \dq F07\dq~und \dq NF03 \dq. Anforderung F07 ist nur teilweise erfüllt, da im Rahmen des Proof of Concept der Bereich der Spannungsversorgung zwar geplant, jedoch nicht implementiert wird. Weiterhin ist im Rahmen des Proof of Concept kein ausführliches Code Review vorgesehen, sodass Anforderung NF03 zwar umgesetzt, die Umsetzung jedoch noch nicht verifiziert ist.
	
\begin{table}[h]
\begin{center}
\begin{tabularx}{\textwidth}{|c|X|c|X|}
\hline
Nr. & Bezeichnung & Status & Bemerkung \\
\hline
F01 &Integration in Automated Testing @ GBC07 & erfüllt &  durch Verwendung von ITE als Sprache sowie entsprechende Architektur \\
\hline
F02 & Schnittstellen Config & erfüllt & Hiperface Funktionalität vollständig implementiert \\
\hline
F03 & Diagnose Funktionalität & erfüllt & Hiperface Funktionalität vollständig implementiert \\
\hline
F04 & Core Funktionalität & erfüllt & Hiperface Funktionalität vollständig implementiert \\
\hline
F05 & Identifikation des DUT & erfüllt & Hiperface Funktionalität vollständig implementiert  \\
\hline
F06 & Hiperface Befehle & erfüllt & Hiperface Funktionalität vollständig implementiert \\
\hline
F07 & Error Codes & teilweise erfüllt & Hiperface Funktionalität vollständig implementiert, Motorfunktionalität vollständig implementiert, Spannungsversorgung geplant \\
\hline


\end{tabularx}
\end{center}
\caption{Abgleich Anforderungen funktional \label{tab:AbgleichF}}
\end{table}



\begin{table}[h]

\begin{center}

\begin{tabularx}{\textwidth}{|c|X|c|X|}

\hline
Nr. & Bezeichnung & Status & Bemerkung \\
\hline
NF01 & Externes Equipment & erfüllt & Einbinden von bisher vorhandenem Equipment geplant und teilweise implementiert, durch geplante Architektur ist das Einbinden weiterer HW möglich \\
\hline
NF02 & Universell einsetzbar & erfüllt & Durch geplante Architektur ist ein Einsatz auch in andern Projekten möglich, die einzelnen Module sind so gekapselt das sie wiederverwendet werden können. \\
\hline
NF03 & Coding Guidelines & teilweise erfüllt & Die Coding Guidelines werden beachtet. Jedoch fand bis jetzt kein entsprechendes Refactoring des Codes statt, um dies zu verifizieren. \\
\hline


\end{tabularx}
\caption{Anforderungen nicht-funktional \label{tab:AbgleichNF}}
\end{center}
\end{table}

	
