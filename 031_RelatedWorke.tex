\onehalfspacing
\section{Related Work}
In diesem Kapitel werden verschiedene Literaturquellen diskutiert, welche im Zusammenhang mit dem Thema der Arbeit stehen. Hierbei wird der Inhalt kurz zusammengefasst und erläutert, in welchem Zusammenhang die Arbeiten stehen.
\subsection*{Optimized test suites for automated testing using different optimization techniques}
In diesem Artikel, welcher von Manju Khari, Prabhat Kumar, Daniel Burgos und Rubén González Crespo geschrieben wurde, befassen sich die Autoren mit der Optimierung von automatisierten Testsuites unter Betrachtung verschiedener Optimierungstechniken. Hierbei werden Tools zur automatischen Generierung von Testsuites verglichen. Die Ergebnisse werden dann im Kontext des Automated Testing betrachtet. Die Arbeit zeigt auf, welche Möglichkeiten bei der Optimierung und automatischen Generierung von Softwaretests bestehen. Im Zuge des konkreten Projekts wird ebenfalls eine Testsuite erstellt. Die Betrachtung von eventuellen Optimierungsmöglichkeiten ist für den späteren Verlauf des Projekts durchaus von Interesse.\cite{ManjuKhariPrabhatKumarDanielBurgosRubenGonzalezCrespo.2017}

\subsection*{Eine Technologe für das durchgängige und automatisierte testen eingebetteter Software}
Die Arbeit, welche durch Dipl.-Inform. Till Fischer im Rahmen seiner Dissertation an der Fakultät für Elektrotechnik und Informationstechnik des Karlsruher Instituts für Technologie (KIT)durchgeführt wurde, werden neben den Grundlagen des Testens von eingebetteten Systemen auch die verschiedenen Testebenen diskutiert. Ziel der Arbeit ist die Verbesserung der Durchgängigkeit des Testprozesses für eingebettete Systeme. Als Lösungsstrategie wird durch Fischer eine Testlösung versiert, welche den Quelltext der ausgeführten Software, die Netzwerkkommunikation, sowie das physikalische Verhalten an elektrischen Schnittstellen und der simulierten Umgebung abdeckt. Hier ist der Bezug zum in dieser Arbeit behandelten Automated Testing @ GBC07 Projekt zu erkennen. \cite{Dipl.Inform.TillFischer.2016}


